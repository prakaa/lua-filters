% Options for packages loaded elsewhere
\PassOptionsToPackage{unicode}{hyperref}
\PassOptionsToPackage{hyphens}{url}
%
\documentclass[
]{article}
\usepackage{amsmath,amssymb}
\usepackage{iftex}
\ifPDFTeX
  \usepackage[T1]{fontenc}
  \usepackage[utf8]{inputenc}
  \usepackage{textcomp} % provide euro and other symbols
\else % if luatex or xetex
  \usepackage{unicode-math} % this also loads fontspec
  \defaultfontfeatures{Scale=MatchLowercase}
  \defaultfontfeatures[\rmfamily]{Ligatures=TeX,Scale=1}
\fi
\usepackage{lmodern}
\ifPDFTeX\else
  % xetex/luatex font selection
\fi
% Use upquote if available, for straight quotes in verbatim environments
\IfFileExists{upquote.sty}{\usepackage{upquote}}{}
\IfFileExists{microtype.sty}{% use microtype if available
  \usepackage[]{microtype}
  \UseMicrotypeSet[protrusion]{basicmath} % disable protrusion for tt fonts
}{}
\makeatletter
\@ifundefined{KOMAClassName}{% if non-KOMA class
  \IfFileExists{parskip.sty}{%
    \usepackage{parskip}
  }{% else
    \setlength{\parindent}{0pt}
    \setlength{\parskip}{6pt plus 2pt minus 1pt}}
}{% if KOMA class
  \KOMAoptions{parskip=half}}
\makeatother
\usepackage{xcolor}
\usepackage{longtable,booktabs,array}
\usepackage{calc} % for calculating minipage widths
% Correct order of tables after \paragraph or \subparagraph
\usepackage{etoolbox}
\makeatletter
\patchcmd\longtable{\par}{\if@noskipsec\mbox{}\fi\par}{}{}
\makeatother
% Allow footnotes in longtable head/foot
\IfFileExists{footnotehyper.sty}{\usepackage{footnotehyper}}{\usepackage{footnote}}
\makesavenoteenv{longtable}
\usepackage{graphicx}
\makeatletter
\def\maxwidth{\ifdim\Gin@nat@width>\linewidth\linewidth\else\Gin@nat@width\fi}
\def\maxheight{\ifdim\Gin@nat@height>\textheight\textheight\else\Gin@nat@height\fi}
\makeatother
% Scale images if necessary, so that they will not overflow the page
% margins by default, and it is still possible to overwrite the defaults
% using explicit options in \includegraphics[width, height, ...]{}
\setkeys{Gin}{width=\maxwidth,height=\maxheight,keepaspectratio}
% Set default figure placement to htbp
\makeatletter
\def\fps@figure{htbp}
\makeatother
\setlength{\emergencystretch}{3em} % prevent overfull lines
\providecommand{\tightlist}{%
  \setlength{\itemsep}{0pt}\setlength{\parskip}{0pt}}
\setcounter{secnumdepth}{-\maxdimen} % remove section numbering
\makeatletter
\@ifpackageloaded{subfig}{}{\usepackage{subfig}}
\@ifpackageloaded{caption}{}{\usepackage{caption}}
\captionsetup[subfloat]{margin=0.5em}
\AtBeginDocument{%
\renewcommand*\figurename{Figure}
\renewcommand*\tablename{Table}
}
\AtBeginDocument{%
\renewcommand*\listfigurename{List of Figures}
\renewcommand*\listtablename{List of Tables}
}
\newcounter{pandoccrossref@subfigures@footnote@counter}
\newenvironment{pandoccrossrefsubfigures}{%
\setcounter{pandoccrossref@subfigures@footnote@counter}{0}
\begin{figure}\centering%
\gdef\global@pandoccrossref@subfigures@footnotes{}%
\DeclareRobustCommand{\footnote}[1]{\footnotemark%
\stepcounter{pandoccrossref@subfigures@footnote@counter}%
\ifx\global@pandoccrossref@subfigures@footnotes\empty%
\gdef\global@pandoccrossref@subfigures@footnotes{{##1}}%
\else%
\g@addto@macro\global@pandoccrossref@subfigures@footnotes{, {##1}}%
\fi}}%
{\end{figure}%
\addtocounter{footnote}{-\value{pandoccrossref@subfigures@footnote@counter}}
\@for\f:=\global@pandoccrossref@subfigures@footnotes\do{\stepcounter{footnote}\footnotetext{\f}}%
\gdef\global@pandoccrossref@subfigures@footnotes{}}
\@ifpackageloaded{float}{}{\usepackage{float}}
\floatstyle{ruled}
\@ifundefined{c@chapter}{\newfloat{codelisting}{h}{lop}}{\newfloat{codelisting}{h}{lop}[chapter]}
\floatname{codelisting}{Listing}
\newcommand*\listoflistings{\listof{codelisting}{List of Listings}}
\makeatother
\ifLuaTeX
  \usepackage{selnolig}  % disable illegal ligatures
\fi
\IfFileExists{bookmark.sty}{\usepackage{bookmark}}{\usepackage{hyperref}}
\IfFileExists{xurl.sty}{\usepackage{xurl}}{} % add URL line breaks if available
\urlstyle{same}
\hypersetup{
  pdftitle={short-captions.lua},
  hidelinks,
  pdfcreator={LaTeX via pandoc}}

\title{short-captions.lua}
\author{}
\date{}

\begin{document}
\maketitle

\listoffigures
\hypertarget{short-captions-in-output}{%
\section{\texorpdfstring{Short captions in
\LaTeX~output}{Short captions in ~output}}\label{short-captions-in-output}}

For latex output, this filter uses the attribute \texttt{short-caption}
for figures so that the attribute value appears in the List of Figures,
if one is desired.

\hypertarget{usage}{%
\section{Usage}\label{usage}}

Where you would have a figure in, say, markdown as

\begin{verbatim}
![The caption](foo.png ) 
\end{verbatim}

You can now specify the figure as

\begin{verbatim}
![The long caption](foo.png){short-caption="a short caption"} 
\end{verbatim}

If the document metadata includes \texttt{lof:true}, then the List of
Figures will use the short caption. This is particularly useful for
students writing dissertations, who often have to include a List of
Figures in the front matter, but where figure captions themselves can be
quite lengthy.

\begin{verbatim}
pandoc --lua-filter=short-captions.lua article.md -o article.tex

pandoc --lua-filter=short-captions.lua article.md -o article.pdf
\end{verbatim}

\hypertarget{example}{%
\section{Example}\label{example}}

Fig.~\ref{fig:shortcap} is an interesting figure with a long caption,
but a short caption in the List of Figures.

\begin{figure}
\hypertarget{fig:shortcap}{%
\centering
\includegraphics[width=0.5\textwidth,height=\textheight]{fig.pdf}
\caption[{A short caption with math \(x^n + y^n = z^n\)}]{This is an
\emph{extremely} interesting figure that has a lot of detail I will need
to describe in a few sentences. This figure has a short caption that
will appear in the list of figures. Other attributes are
preserved}\label{fig:shortcap}
}
\end{figure}

\hypertarget{limitations}{%
\section{Limitations}\label{limitations}}

\begin{itemize}
\tightlist
\item
  The filter will process the \texttt{short-caption} attribute value as
  pandoc markdown, regardless of the input format.
\item
  It does not work for tables and listings yet.
\item
  But it works with pandoc-crossref, regardless of the order of
  application.
\end{itemize}

\end{document}
